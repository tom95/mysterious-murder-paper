\section{Evaluation} \label{evaulation}

\subsection{Linear Logic}
While linear logic is more suited to represent real-life situations than classical logic, it showed shortcomings when applied to our domain.
In linear logic all rules are deterministic, meaning every rule has exactly one possible outcome.
There is no possibility for chance.
This forced us to have multiple rules to represent possible different resolutions to some rules.
One example is the rule \lstinline{steal}, which allows an actor to attempt to steal money from another.
There were two possibilities when using this rule: the actor could be successful at stealing or he or she could be caught.
These possibilities should obviously have a different effect on the state of the simulation.
This kind of non-determinism is not supported by linear logic.
To combat this we introduced a random component into our linear logic implementation that could pick one of the outcomes based on a probability.\todo{adapt this or approach about non-determinism}


Another drawback of linear logic was that it only knows the resources that exist in the current state.
There is no possibility to test for the absence of a given resource.
While this did not pose a significant problem, it made the setup a lot more complex, both in terms of authorship and in terms of computation.
For many predicates we now had to introduce an opposite and manually ensure that the resources always appeared mutually exclusive.

For example our system kept track of which characters in the game were related to each other via the predicate \lstinline{related Character Character'}.
When we introduced a new rule \lstinline{get_married}, we wanted to check that the characters getting married were not related to each other.
As linear logic does not allow this, we had to introduce a new predicate \lstinline{not_related} and make sure to assign exactly one of these two predicated during the generation of the initial state.

These were two reasons why the rules tended to be long and complex.\todo{is this here to stay?}

\todo{talk a bit about the degree of responsibility when defining the rules of society. maybe mention a balance between what a player can understand based on their own societal experience (which is a lot of what detective work is about after all, right?) and what the author thinks a society should act like}

\subsection{Authorship and Rule Set}
\todo{lack of time, not complex enough, not enough variability, sign that base system should be more flexible, authors not in IT may find linear logic confusing, missing unique, complex reasoning / story, flavouring}