\section{Background} \label{background}

\subsection{Linear Logic: Ceptre} \label{linear_logic}

Ceptre is an implementation of a linear logic system, providing a rule specification language to allow authors the formalization of agent interactions and gameplay rules and their subsequent simulation.

In Ceptre, a \enquote{game} consists of terms, predicates, rules, and an initial configuration.

The game designer may \todo{essentially repeat abstract and focus on parts important to us}

Linear logic is a formal logic that extends the classical logic used in mathematics.
The classical logic concerns itself only with \enquote{stable truths}.
These can, however, not always be applied directly to real-life situations, as they lack causality \cite{girard_1995}.
While in classical logic truth all conditions stay the same after an evaluation, in reality the conditions may be modified.
For example, trading money for an object removes the possibility of spending this money again and changes the truths \enquote{I have money} to \enquote{I have this object}.
Linear logic focuses on its formulas as resources, rather than truths \cite{sep-logic-linear}.
New resources may be added to the simulation's state or removed in the course of rule evaluations.

The application of linear logic has been studied in the context of games and programming languages, as well as in quantum physics and linguistics \cite{sep-logic-linear}. \todo{(is this relevant? if so cite)}

Ceptre allows a game designer to specify terms, rules and predicates as well as an initial configuration S.
The initial configuration S consists of a set of predicates that represent the collection of truths at this point, such as \enquote{I have money}.
The execution of a Ceptre simulation consists of a series of repeated steps.
During each step, it will first be determined which rules can be applied in accord with the current state.
A rule may have a set of preconditions, which is the resources that will be consumed from the state if that rule is applied, and a set of resulting resources.
This corresponds to the possible actions an agent could take or in which a simulation may evolve.

A valid rule can either be chosen interactively by a player, or randomly. 
The chosen rule will then be applied to the state, creating a new instance of state S', on which the simulation step is then repeated until some terminal condition is reached, for example no more rules being available.
