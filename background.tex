\section{Background} \label{background}

\subsection{Linear Logic: Ceptre} \label{linear_logic}

Ceptre is an implementation of a linear logic system (??) that uses resources.

Linear logic is a formal logic that extends the classical logic used in mathematics.
This classical logic concerns itself only with "stable truths".
These can, however, not always be applied directly to real-life situations, as they lack causality \cite{girard_1995}.
While in classical logic truth all conditions stay the same after an evaluation, in reality the conditions may be modified.
For example trading money for an object removes the possibility of spending this money again.
Linear logic focuses on its formulas as resources, rather than "truth" \cite{sep-logic-linear}.
These resources may be manipulated in the course of rule evaluations.

The application of linear logic has been studied in the context of games and programming languages, as well as quantum physics and linguistics \cite{sep-logic-linear}. \todo{(is this relevant? if so cite)}

Ceptre allows a game designer to specify terms, rules and predicates as well as an initial configuration.
The initial configuration S consists of a set of predicates that represent the "truth" at this point.
The execution of a Ceptre program consists of a series of repeated steps.
During each of those repetitions, it must first be determined which rules can be applied in accord with the current state.
This corresponds to the possible actions a character could take.
A valid rule can now be chosen either by a player or randomly. 
This rule will then by applied to the state, creating a new instance S'.
This state S' may have additional predicates or may be missing predicates that became false after the application of the rule (as per the definition of linear logic).

