\section{Introduction} \label{introduction}

In this paper we will discuss an approach to create murder mystery stories via procedural generation
We will discuss several artificial intelligence (AI) approaches we employed.

\subsection{Murder Mysteries}

Murder mysteries as a genre first became popular in the 19th century as a literary genre.
They quickly grew in popularity.
These detective stories reached their height of popularity in the 1920s and 30s, which has since been referred to as the Golden Age of Detective fiction.
Many authors from that time are still prevalent in today's pop culture, as can be seen by for example the 2017 adaptation of Agatha Christie's novel Murder on the Orient Express.
Novels from that era often followed very strict rules as defined by Knock's "Ten Commandments".\todo{ref}

Today, murder mysteries have found their way into most genres.
Television series and movies especially feature them in many different variations.
\todo{today popular in movies television (45?) general public?, find stats}

These crime shows come in a huge variation of sub-genres, be it the classical detective story or a more outlandish settings, such as ... \todo{mention stuff like izombie or lucifer}.

The board game Cluedo is another representation of this genre, letting the players compete on who can solve the murder of Dr. Black the fastest.

\subsection{Procedural Story Generation}
Rather than replacing authors by generating stories automatically, this project aims to provide a tool for authors to craft rules that inform the mechanics of a world.
These mechanics, combined with procedural generation, then yield 
