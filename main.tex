\documentclass[sigchi,review]{acmart}

\usepackage{booktabs} % For formal tables


% Copyright
%\setcopyright{none}
%\setcopyright{acmcopyright}
%\setcopyright{acmlicensed}
%\setcopyright{rightsretained}
%\setcopyright{usgov}
%\setcopyright{usgovmixed}
%\setcopyright{cagov}
%\setcopyright{licensedcagov}
%\setcopyright{cagovmixed}
%\setcopyright{licensedothergov}

\copyrightyear{2018}

\usepackage[T1]{fontenc}
\usepackage{todonotes}
\usepackage{csquotes}
\usepackage{listings}

\def\UrlBreaks{\do\/\do-}

\begin{document}
\title{Mysterious Murder -\\
MCTS-driven Murder Mystery Generation}
%\subtitle{Extended Abstract}

\author{Corinna Jaschek}
\affiliation{%
  \institution{Hasso Plattner Institute}
  \streetaddress{Prof.-Dr.-Helmert-Stra{\ss}e 2-3}
  \city{Potsdam}
  \state{Brandenburg}
  \postcode{14482}
}
\email{corinna.jaschek@student.hpi.de}

\author{Tom Beckmann}
\affiliation{%
  \institution{Hasso Plattner Institute}
  \streetaddress{Prof.-Dr.-Helmert-Stra{\ss}e 2-3}
  \city{Potsdam}
  \state{Germany}
  \postcode{14482}
}
\email{tom.beckmann@student.hpi.de}

\author{Jaime A. Garcia}
\affiliation{%
  \institution{University of Technology Sydney}
  \streetaddress{15 Broadway}
  \city{Sydney}
  \state{Australia}
  \postcode{2007}
}
\email{jaime.garcia@uts.edu.au}

\author{William L. Raffe}
\affiliation{%
  \institution{University of Technology Sydney}
  \streetaddress{15 Broadway}
  \city{Sydney}
  \state{Australia}
  \postcode{2007}
}
\email{william.raffe@uts.edu.au}

% The default list of authors is too long for headers.
\renewcommand{\shortauthors}{C. Jaschek, T. Beckmann, J. Garcia, W. Raffe}


\begin{abstract}
  We present an approach to procedurally generate the narrative of a simple murder mystery.
  To do so, we use the rule evaluation system as implemented by \emph{Ceptre}, which employs linear logic to resolve valid actions during each step of the simulation.
  We extend this system with a concept of believable agents to make consecutive actions appear to have a causal connection, so that players can comprehend the flow of events.
  The parts of the generated narratives are then presented to a player whose task it is to figure out who the murderer in this story could have been.

  Rather than aiming to replace highly authored narratives, this project generates puzzles, which may contain emerging arcs of story that involve the player more strongly in the simulation.
  While we found that even a simple rule set can create stories that are interesting to reason about, we assume that this type of system is flexible enough to create considerably more engaging stories if enough time is invested in authoring them.
\end{abstract}

%
% The code below should be generated by the tool at
% http://dl.acm.org/ccs.cfm
% Please copy and paste the code instead of the example below.
%
\begin{CCSXML}
<ccs2012>
 <concept>
  <concept_id>10010520.10010553.10010562</concept_id>
  <concept_desc>Computer systems organization~Embedded systems</concept_desc>
  <concept_significance>500</concept_significance>
 </concept>
 <concept>
  <concept_id>10010520.10010575.10010755</concept_id>
  <concept_desc>Computer systems organization~Redundancy</concept_desc>
  <concept_significance>300</concept_significance>
 </concept>
 <concept>
  <concept_id>10010520.10010553.10010554</concept_id>
  <concept_desc>Computer systems organization~Robotics</concept_desc>
  <concept_significance>100</concept_significance>
 </concept>
 <concept>
  <concept_id>10003033.10003083.10003095</concept_id>
  <concept_desc>Networks~Network reliability</concept_desc>
  <concept_significance>100</concept_significance>
 </concept>
</ccs2012>
\end{CCSXML}

\ccsdesc[500]{Computer systems organization~Embedded systems}
\ccsdesc[300]{Computer systems organization~Redundancy}
\ccsdesc{Computer systems organization~Robotics}
\ccsdesc[100]{Networks~Network reliability}


\keywords{}

\maketitle

\section{Introduction} \label{introduction}

Murder mysteries as a genre first became popular in the 19th century in literature.
They then quickly grew in popularity over the next decades.
These detective stories reached their height of popularity in the 1920s and 30s, which has since been referred to as the Golden Age of Detective fiction.
Many authors from that time are still prevalent in today's pop culture, as can be seen by for example the 2017 movie adaptation of Agatha Christie's novel \emph{Murder on the Orient Express}.
Novels from that era often follow very strict rules as defined for example by Van Dine's \emph{Twenty Rules for Writing Detective Stories} \cite{van_dine_1928} or Knox' \emph{Ten Commandments} \cite{knox_1929}.

Today, murder mysteries have found their way into most genres.
Television series and movies feature them especially often.
These crime shows come in a huge variation of sub-genres, be it the classical detective story or a more outlandish setting, whether it is a Zombie helping to solve crime by eating the victims brains\footnote{\url{https://www.imdb.com/title/tt3501584/?ref_=nv_sr_1}} or the devil himself coming to earth to find who did it\footnote{\url{https://www.imdb.com/title/tt4052886/?ref_=fn_al_tt_1}}.
The board game Cluedo is another representation of this genre, letting the players compete on who can solve the murder of Dr. Black the fastest\footnote{\url{https://shop.hasbro.com/en-us/product/clue-game-2013-edition:9BCB9C5E-5056-9047-F505-6D1166C5961E}}.

\subsection{Procedural Murder Mystery Generation}
Murder mysteries in video games often follow a linear, pre-written path that may include branching options.
The extent to which games divert from this mode of storytelling will be explored in Section \ref{related_work}.
Rather than replacing authors by generating stories automatically, this project aims to provide a tool for authors to craft rules that inform the mechanics of a world.
These mechanics, combined with a procedural simulation, then yield a generated narrative whose quality will largely depend on the complexity of the pre-authored rules.
Using this system instead of a branching narrative ensures a higher degree of variability and the possibility of emergent, perceived creativity in the stories on each playthrough, as the procedural generation system produces elements that the player believes to recognize.
The stories generated in this way lend themselves better to be displayed in an amusing puzzle setting, rather than a highly dramatic story that is authored to provide the best pacing and suspense.

Our system uses Ceptre as a basis. Ceptre is a system to formulate and run interactive narratives or simulations \cite{martens_2015}.
We extend Ceptre's core concept with non-deterministic outcomes and a reward system for actions to drive artificial agents.
Our simulation is neither driven by a human player or random choices, which are the two options Ceptre offers.
Instead, we implement MCTS agents that drive the story by evaluating the possible options to maximize their reward.
To make predicting the outcome of the simulation multiple steps ahead feasible in terms of performance, we re-implemented Ceptre's engine in Python.

In Section \ref{related_work} we will introduce related work in the fields of murder mysteries in video games, procedural story generation, and the story generation system \emph{Ceptre}.
Section \ref{approach} will explain our approach to our own linear logic evaluator, our rule set, believable agent system, and the game's user interface.
The Section \ref{evaulation} discusses, in particular, our opinion on linear logic in this system and the challenges coming with the authorship of the rules.
Sections \ref{future_work} and \ref{conclusion} will talk about possible next steps that would drive this project further and provide a summary of our findings.

\section{Related Work} \label{related_work}

In this section, we will give an overview on existing murder mystery video games and examine how they approach the various challenges of the genre, survey projects that have used procedural generation in the context of game narratives and present the rule-specification language Ceptre created by Martens, which our own system is based on.

\subsection{Murder Mystery Video Games}

While prevalent in many genres, as seen in section \ref{introduction}, murder mysteries have only been explored in video games rarely, when compared to other genres.
We put forward the claim that this is due to the complexity involved in detective work as portrayed by literature and film.
Most murder mystery games rely on handwritten narratives that are similar in plot to a novel or movie.
The player has the opportunity to follow the detectives along the story and gather evidence, interview suspects, or follow leads.
In some cases they may also expose lies and find holes in presented testimonies.
However, most games do not allow the players to make connections between the evidence and testimonies and let them come to their own conclusions. With most steps represented as multiple choice questions, the players just have to pick the correct choice or guess until they get it right \cite{brown_2017}. 

This method is employed in games like \emph{LA Noire}, where falsely accusing suspects will make them offer less information to the player, but will never have game-changing consequences.
This ensures that players will always be able to complete the game; they will always be led to reach the next scripted story point in the branching or linear narrative, even if they took wrong decisions before.

While this is a valid choice, it has been noted that the resulting game may be less compelling.
When the answers are all given within the game, the player cannot become fully immersed in playing the clever detective \cite{goldstein_2018}.

However, when players are left on their own to collect all evidence, it can lead to games like \emph{Murder on the Mississippi}, which becomes unsolvable if players miss a piece of information early on in the game \cite{petit_2018}.

To make sure no clues can be missed while still allowing for more interaction than simple multiple choice options, games often feature in-game notebooks or a collection of relevant items and notes.
For example in the game \emph{Discworld Noir} or... \todo{write on!}
These written down clues often spoil the puzzles they represent as they need to feature the important aspect of the clue in order to be useful at all.

Some games have attempted to remove this step by letting players take their own notes and offering other option to receive more information, such as the in-game search engine in \emph{The Shivah} or the telephone book in \emph{Sherlock Holmes: Consulting Detective}, which only when used correctly open up potential new options for interaction.
\todo{Why is this not ideal?}
The game \emph{Her Story} is an example of a completely different way of approaching this problem.
The murder mystery game does not involve any sort of confrontation of the suspects.
In fact, there is no in-game way of determining if you have come to the right conclusion at all.
Players simply gather information by searching a police database until they are satisfied with their understanding of the events.

We claim that reaching the correct solutions in this manner tends to be more satisfying to players, as it is a testament to their skills.

Yet, no matter to which of these categories a game belongs, it has a relatively low replay value, as once the murderer is known, the main motivation for playing the game ceases to exist.

\emph{ClueGen} attempts to rectify this by using procedural generation to the plots of murder mysteries.
A number of possible motives and histories are used to create a multitude of narratives.
The addition of possible red herrings or lying non-player characters enhances the complexity even further.
Even with a relatively small set of predefined options the resulting mysteries were sufficiently interesting to players.
The players recognized complex, underlying intentions and structures, even in cases where these were not intended by the author of the rules informing the procedural generation program \cite{stockdale_2016}.

\subsection{Procedural Narrative Generation}

For the game PromWeek, McCoy et al created a \enquote{model for socially-oriented gameplay} named \emph{Comme il Faut}.
The model draws from a corpus of social norms that inform the decisions of AI characters each turn of the game.
It then generates animations to visualize to players how characters react to their decisions.
To reach this point, PromWeek required two years of manual authoring to embed social biases and conventions in a formalized rule system \cite{mccoy_2013, McCoy:2011:CIF:3014589.3014617}.

A similar demonstration of an engaging, fully realized social game is Fa\c{c}ade.
Here, as well, it took the authors multiple years to create the necessary rules.
Instead of leaning heavily on exclusively generated content, Facade chose to define story events that may happen at any time if its preconditions are fulfilled, dynamically influencing the story and shaping it as much as possible based on the player's interaction with the game \cite{Mateas_2003}.

Rowe et al demonstrate the use of reinforcement learning to tweak stories.
A narrative planning system was trained by having a large number of pupils play an educative story-based game and fill out surveys on their experience.
Different branching story options were selected for different players.
This allowed an automated analysis of the story variants that were particularly well received and had the highest impact on the education of the players \cite{rowe_2014}.

\emph{Ceptre}\footnote{https://github.com/chrisamaphone/interactive-lp} is a \enquote{rule specification language} for interactive narratives or simulations \cite{martens_2015}.
Its intention is to allow rapid prototyping by providing a language specific to this domain.
It uses linear logic as the foundation for the interpretation, which will be explained in more detail in section \ref{linear_logic}.

\section{Approach} \label{approach}

In this section, we will outline our approach to a linear logic evaluator, describe how we used to it to generate stories in the domain of murder mysteries, and finally describe the user interface we built to allow players to interact with the system.

\subsection{Linear Logic Evaluator}

Our approach on story generation is based on the approach used by the Ceptre story generation engine.
This engine comes with a linear logic evaluator that takes rules that describe the transformation of resources to model the development of a story.
The Ceptre engine is written in the Standard Meta Language.
Early in the project, we had to realize that we were using too many constraints, modeled as resources, for Ceptre to handle.
Execution speed would quickly deteriorate, especially as the simulation dragged on.
As a consequence, we decided to implement a version of Ceptre, reduced to the aspects that we were making use of.
This meant that we dropped features like Ceptre's stages and its advanced type system with support for inheritance.
The type system, which allowed for elegant rules like \enquote{has Character gun} for the predicate \enquote{has character object} and the subtype \enquote{gun} of \enquote{object}, made the evaluation considerably more complex as each type instance would increase the possible permutations of each rule making use of it.
For our simple use cases, we were able to avoid using types other than our actors by using rules of the form \enquote{has\_gun Character}.

Our linear logic evaluator takes a set of actors, a set of rules and a set of predicate instances that form the initial state of the simulation.
Predicates have a name and an ordered list of placeholders that can be taken up by any actors.
A predicate instance is a predicate of which all placeholders have been configured with concrete actors.
Rules consist of a left-hand side and a right-hand side, both containing predicates.
The predicate's placeholders are numbered consistently, so that advanced rules that refer to the same actors across the predicates of a rule are possible.
To this extent, we form a strict subset of Ceptre's own evaluator. 

We extended this system by introducing non-determinism. This means that the right-hand side may list multiple sets of predicates that are all possible outcomes of activating this rule, together with a probability of each outcome occurring. Prior to this, we would have two rules for each possible outcome, for example one for stealing from another actor successfully and another for being caught. This worked fine while randomly selecting rules, but as soon as we used more sophisticated artificial intelligence, the agents figured out that picking the rule where they get caught was not a good choice.

To further guide the artificial intelligence in its decisions, we added a system of rewards to each rule. For each outcome of a rule, the agent would receive a reward or a penalty in different categories.
As categories, we chose \emph{social}, \emph{sanity} and \emph{fulfillment}, loosely based on the top three layers of Maslow's hierarchy of needs \todo{reference?, correct names of layers if necessary}, \emph{belongingness and love}, \emph{esteem} and \emph{self-actualization}.
These rewards proved to be general enough so that each rule we developed could have an intuitive value associated with it.
Lastly, we added a built-in templating system that used knowledge embedded in the system to formulate the rules as natural sentences.
This allowed the formulation of templates like this:,
\begin{lstlisting}
Shocked by the revelation that
[0:his|her] [2:husband|wife] was
cheating, {0} murdered [2:his|her]
lover {1}.
\end{lstlisting}
where the each \lstinline|{i}| will be replaced with the name of the i-th actor in the rule and each \lstinline{[i:a|b]} will be replaced by a or b depending on if i-th actor is male or female.

\subsection{A Murder Mystery Rule Set}

As a basis for the rules, we chose to introduce resources that model the current relationships between characters and general tendencies in their character to act in certain ways. For these, we adapted the \emph{Big Five Personality Traits} \cite{rothmann_coetzer_2003}, giving each character a balance of three resources for each opposing pair of traits. For example, a character might receive two \emph{cautious} resources and one \emph{curious} resource, or three \emph{confident} resources and no \emph{insecure} resources. This allowed authoring rules that require a certain tendency towards one of the extremes of each trait pair. To model the relationships between characters, we used Plutchik's Wheel of Emotion\todo{ref}. Again, we have opposing pairs that are however not necessarily exclusive in our system. Spending time with another actor may yield \emph{trusting} resources between the two, while fighting might yield \emph{anger} resources. A \emph{making up} rule would then allow removing \emph{anger} resources again. To model more extreme characters, we introduced a system similar to alignments as found in some role-playing games. Actors may either be good, neutral or, evil, which will enable them to pick certain options more easily than other characters.

In the next step, we generate relationships and attributes relevant to our domain of classical murder mystery \todo{check term classical}. This includes whether a character is rich (\enquote{has\_money Character}), their current relationship to other characters using the wheel of emotion, or whether the characters are either related, or lovers or married.

Our rule authoring system is conceptually identical to that used in Ceptre. We, however, directly compose Python objects, which has the disadvantage of having to adhere to Python's syntax rather than being able to define a domain specific language, but the advantage that we can define functions that group often used sets of predicates, enable domain specific consistency checks, or toggle groups of rules with \enquote{if}-conditions. All of this would have of course also been able to in Ceptre, but would have required introducing new syntactical elements in the Ceptre language, while we effectively work on a meta language level at all times.

Most of the rules pose a dramatic conflict between characters that generate resources like anger or suspicion.
Typically, there is a countering rule that resolves some of this conflict by removing these resources again.
The system is, however, biased towards escalating the conflict quickly, to ensure that stories do not get out of hand and become too complex to understand for a player. All rules require a motivation as the player would otherwise not have means to follow the flow of events. For example, fighting with another character requires the preexistence of \enquote{anger} between the two. Making up would require characters to have developed two \enquote{anger} between each other before, while there also exists a resource of \enquote{trust} between the two. All rules require certain preconditions, such as characters being alive or dead, or not currently being married when proposing to another character.

In our current rule set, there are three different motives for murder.
Characters that have many \enquote{anger} resources to another character may choose to murder them over that.
Spouses of characters that are cheating may choose to murder their spouse's lover.
Greedy characters may choose to murder a rich character to acquire their money.
All of these rules require the actor to first acquire a weapon, which delays the inevitable at least for one turn, typically making reason and consequence less obvious in the narrative.
Further, we added one rule where characters may be able to kill another character without being in possession of a weapon if they have particularly many \enquote{anger} resources to another character, to make the \enquote{has\_weapon} predicate slightly less of a definitive indicator of a murder.

Other rules to add substance to the narrative include characters attempting to gamble, which may yield debt that motivates them to acquire money.
This can either be done by attempting to steal from other characters, or even by murdering them.
Characters can seduce each other, become lovers or married, or get divorced.
Further, to make the origin of \enquote{anger} resources that play a central role in the murder motives less obvious, we added the concept of spreading the abstract concept of a lie about another character.
In this way, a character X that does not like another character Y can attempt to spread a lie about Y to a character Z.
If the successful outcome is chosen, Z now also becomes angry at Y.
A more sophisticated system may allow characters to construct specific lies, which would add another layer of interest to the game, as characters would then be able to formulate testimonies which they believe to be true based on information they received from another character, but which the player can then identify as lies.

We added non determinism wherever possible, and in particular to those rules with high rewards for a positive outcome, to ensure that he artificial intelligence will choose options that may have a high pay off for it, but may also result in dramatic tension, as for example, one character catches another in an attempt to steal from them.

Having only few motives for murder does in fact align with the most common structure of Agatha Christie's novels, which we took as a major influence.
Typically, the reason for murder is one of those we listed above, with the depth of the story coming from the elaborate reasoning and the relationships between the characters.
Putting a focus on delaying the murder for multiple steps is thus important to let agents interact and provide variability to the otherwise few motives.

\subsection{Believable Agents}
In the first iteration of the prototype, we used agents that took actions randomly. This resulted in stories that were varied, but were hard to follow, as no higher level intention of the agents became apparent in the story, only by chance.

To improve the believability of the flow of actions, we added certain rewards to each action that an agent can take. Some actions may incur a penalty, like picking up a weapon which decreases the agent's sanity score, but ultimately may allow it to commit a murder for a reward in fulfilment, if the agent has a fitting motive. The agent's actions became a lot more coherent in this way, as they follow the directions as given by the authored reward structures. With an ability to plan ahead for multiple steps, different agents may try to anger characters about others or find different ways to fulfil the preconditions of the rules with the highest rewards. 

To enable this sort of planned behavior, we used Monte Carlo Tree Search (MCTS). We pass the simulation to a MCTS agent, which then expands child states via UCB1 \todo{ref and is it actually called that?}. Since the simulation has no clear terminal state, we let the simulation play out a fixed number of steps every time instead. Considering a murder a terminal would have been an option, but this might actually discourage murders from happening, as actors who do not commit a murder get to collect rewards from more steps during playout. An alternative to this model would have been to only determine win/lose per playout, ending a playout with \enquote{win} if the character commits a murder and \enquote{lose} if they fall victim to one. This, however, would discourage actors from considering all other options that would otherwise bring them smaller rewards.

From experimentation, good values for the maximum number of expanded states is around 100, while the number of playout steps if limited to 10. Most of the time, simulations will be shorter than 30 steps, making 10 a good balance between performance and allowing actors a good amount of foresight. The accuracy of the MCTS agents is diminished by the non deterministic choices. While expanding states, the actor will only get to know a single of those states and assume that this is the outcome. This results in sub optimal actors that sometimes act more pessimistic or optimistic when it comes to certain decisions, for example if they should propose to someone. As such, we found that not taking another step of optimization to enable the actors to explore all possible outcomes of actions they take, results in a more believable behavior.

With the MCTS agents, we hit significant performance problems as we now had to evaluate valid actions at the current state 300 times more often, with 30 expansions and 10 playout steps. Using pre-calculated hashes for all lookups into the resource state allowed the simulation to progress reasonably fast, at about a pace that allowed a human to follow the event flow as decisions are made. Typically, evaluating MCTS actors is a very fast option \todo{ref for where they say it's fast}, but this is based on the assumption that calculating rewards is the expensive actions, while the random playout afterwards happens for free. For our simulation, however, each step of the simulation is 

On our way to the MCTS agents, we first built an agent system based on OpenAI's \emph{deepq} Q-learning algorithm \todo{ref}. We set up the state space to reflect all available actions flattened as a two-dimensional array of integers, where each row's first integer corresponds to the constant index of a rule and each column then corresponds to the constant index of a character. A zero in either meant that this rule was unset or this rule did not involve all three characters. The action space then provided the agent with a choice of any of these rules by picking its integer index in the state space array. The challenge that we encountered with this setup primarily comes from the high variability in quantity of the available rules. We put a high negative reward on picking an invalid choice, but, even with a high factor of exploration, only made the agent learn to favor picking the first couple of options that it learned were typically safe. In an attempt to improve this situation, we shuffled the array's rows, such that invalid rules could be found at any index. Here, training over more than three hours did not lead to the agent reliably picking valid rules, which led us to the assumption that the state space was too large to figure out on an automatic basis. More specifically, we offered the agent a state space of 20 rows, with 4 columns each to host the rule index and up to three involved actors. The actor index ranged from 0 to 4 to mark the invalid actor and the 4 valid actor indices and the range of the rule index depended on how many rules the author supplies to the system, which in our case was around 20.

To mention one more attempt we explored, with the motivation of creating a semi-automatic authoring system for rules, we employed word2vec \todo{ref} as a form of completion dialog as found in modern integrated development environments.
Indeed, when tasked with solving word vectors for \enquote{gun} minus \enquote{kill} plus \enquote{cook}, the word2vec system yielded words such as \enquote{wok} or \enquote{dishwasher}.
We used the pre-trained \todo{dataset name here and who provides it from where}.
In an attempt to have word pairs closer to our domain, we trained our own model with murder mystery stories in the public domain, in our case various Sherlock Holmes stories by Sir Arthur Conan Doyle, as well as early stories by Agatha Christie.

\subsection{User Interface} \label{user_interface}

We used the Python library Kivy to develop our user interface (UI).
This allowed us to directly use the Python objects that were created during the murder mystery generation, without the need for serialization when crossing language boundaries.
One important goal during the UI development was to have a very visual representation of the state that is easy to understand for players.
To achieve this, we relied on images to display our characters, objects and rules.
In the beginning we relied on a very small set of sample images for our actors, which quickly led to confusion as the same portrait was used for multiple characters.
Instead we started using emojis as our portraits and other images.
This allowed us to have a great range of portraits that represent different genders, skin colours, ages and occupations.
When assigning the images to the characters, we still proceed mostly randomly, only differentiating between male and female characters.
However, with more than 350 possible images to choose from it is very rare that two characters will look the same.

Emojis also allowed us to have uniformity in our other symbols.
Apart from the characters we also use the emojis to represent the emotions and state of relationship between characters.
This has worked very well with using emojis, as they were created to convey these emotions and are well known by the majority of people.
This also allows us to communicate the emotions directly without having to use a level of indirection in the form of text.
Especially if a user plays the game repeatedly, it might quickly become boring if always the same text is being used to explain a specific emotion.
Emojis let us avoid this problem, without having to put a lot of effort into writing multiple versions of one statement.
They also help people to quickly recognize the actors in each action, even if they do not know the names.

In addition emojis are used to represent the objects, and the rules which use them.

The rules were harder to represent using emojis.
Some were quite simple, for example murder which we represent simply using a weapon and the two characters involved in the murder.
Others were a bit more complicated, such as the rule 'lie to A about B' which is displayed as the "secret" emoji.
Unfortunately we were not able to find a recognizable emoji for all rules, for example paying a debt to someone or stealing from someone, which are both represented using different money symbols as they are no other, more obvious ones to choose from.

\section{Evaluation} \label{evaulation}

\subsection{Linear Logic}
While linear logic is more suited to represent real-life situations than classical logic, it showed shortcomings when applied to our domain.
In linear logic all rules are deterministic, meaning every rule has exactly one possible outcome.
There is no possibility for chance.
This forced us to have multiple rules to represent possible different resolutions to some rules.
One example is the rule \lstinline{steal}, which allows an actor to attempt to steal money from another.
There were two possibilities when using this rule: the actor could be successful at stealing or he or she could be caught.
These possibilities should obviously have a different effect on the state of the simulation.
This kind of non-determinism is not supported by linear logic.
To combat this we introduced a random component into our linear logic implementation that could pick one of the outcomes based on a probability.\todo{adapt this or approach about non-determinism}


Another drawback of linear logic was that it only knows the resources that exist in the current state.
There is no possibility to test for the absence of a given resource.
While this did not pose a significant problem, it made the setup a lot more complex, both in terms of authorship and in terms of computation.
For many predicates we now had to introduce an opposite and manually ensure that the resources always appeared mutually exclusive.

For example our system kept track of which characters in the game were related to each other via the predicate \lstinline{related Character Character'}.
When we introduced a new rule \lstinline{get_married}, we wanted to check that the characters getting married were not related to each other.
As linear logic does not allow this, we had to introduce a new predicate \lstinline{not_related} and make sure to assign exactly one of these two predicated during the generation of the initial state.

These were two reasons why the rules tended to be long and complex.\todo{is this here to stay?}

\todo{talk a bit about the degree of responsibility when defining the rules of society. maybe mention a balance between what a player can understand based on their own societal experience (which is a lot of what detective work is about after all, right?) and what the author thinks a society should act like}
\section{Future Work} \label{future_work}

While we made significant progress in this project the final outcome is far from being a finished or even playable game.
In this section we will talk about the steps that are necessary before a version of this prototype can be tested by players, such as giving non-playable characters (NPCs) more distinct personalities and the ability to lie, as well as a concept of searching for additional evidence.

\subsection{Enhancing Characters}

The decision to use emojis to represent our characters has served us well, making recognition quick and easy.
The characters themselves are, however, still very rudimentary and lack personality.

We would like to add more options to the characters themselves, such as age and occupation, as well as use the already defined traits during the investigation stage. Trusting characters could reveal information more easily, while others may try to keep secrets.

Using the emojis in a less random and more thought out way would allow us to make relationships between characters more clear. Characters of one family could, for example, have the same skin colour (in most cases).

Currently we only generate whether two characters are related, we do not specify how.
This might be an area where we could improve, which would allow us to introduce more complex rules and allow for more interesting stories, such as rivaling siblings. To make character relations more obvious, we would like to change their last names, so that families and married couples have the same name.

\subsection{Enhancing the Investigation Stage}
To make the game satisfying to players, they should have the option to uncover hidden motives.
Currently none of the actors will attempt to lie to the player, they will not even try to hide information.
While not all actors possess knowledge of everything that has happened, it is only a matter of dedication for the player to question all actors before eventually finding the actor who admits to the murder.

To remedy this situation, we propose a gameplay mechanic, wherein local knowledge exists and certain actors can lie.
More specifically, any actor is allowed to omit information that they consider sensitive.
Only the murderer is allowed to also falsify statements.
For any rule that is taken, uninvolved actors have a chance to witness this action happening.
The probability that this happens is defined per rule, to prevent sensitive information from being witnessed too frequently, such that more realistic testimonies come together.
With these mechanics in place, the game's core objective for players will be to identify the one actor whose testimony contains actual false statements, thereby revealing themselves as the murderer.
This is challenged by the fact that actors the murderer interacted with may omit this interaction, or that the murderer primarily interacted with the victim and no other actors witnessed these interactions.

In a further step, accomplices to the murderer may be defined based on relationships.
The relationship between the murderer and the accomplice may be hidden, the accomplice may be a secret lover for example, or open through family connections or the like.
The accomplice would then also get the option to falsify information in order to cover both of them.
In the hopes that there will be contradictions between the testimony of the murderer and their accomplice, the player would get an additional challenge to find all those involved.

With this model, it is however, questionable if the player will always be able to solve the mystery.
It may very well happen that simply not enough information from uninvolved actors is presented to find contradictions and identify the murderer.
While this of course reflects real life murder mysteries well, it will not lead to satisfying gameplay.
As such, it would be desirable to automatically detect how many statements contradict to filter out potentially unsolvable mysteries.

The current investigation stage could be further improved by not making all actions stop immediately after the murder.
This would make it possible for actors to rid themselves of the murder weapon or grieve for the victim.
Ideally actors would be able to blame the murder on other characters to create scenarios that are more difficult to unravel.

Once these points have been realized, rigorous playtesting would be the next step.
In particular, the number of characters will provide an interesting point of balance.
Having more characters increases the load of information for the player, but also provides them with more points of reference for figuring out contradictions in a testimony and potentially a richer story with more complex character relationships.
Since the interactions between actors are the only way the player gets to know them, giving actors the option to interact more will lead to players identifying actors in richer ways, as already shown in ClueGen \todo{ref}.
An actor may become a lover of particularly many other actors or a family feud may arise.

It also stands to investigate if we can add a mechanic that would create red herrings, or if the procedural generation already lets the player find red herrings without requiring explicit authorship of such rules.

\section{Conclusion} \label{conclusion}

While our murder mystery generator showed promise with regards for variability in the stories it generated, it became clear that this system fundamentally relies on having a well-thought-out set of rules that requires the involvement of an author.
As such, it will not take the place of authored, linear murder mystery games, but rather provide an option to quickly generate ideas for more elaborate plots or generate puzzles with a small scope that a player should be able to solve within 15 minutes.
The explored approaches of combining the linear logic evaluator as implemented by Ceptre with Monte-Carlo-Tree-Search based artificial agents proved to work well to fulfill this purpose.


\bibliographystyle{ACM-Reference-Format}
\bibliography{bibliography}

\listoftodos\relax

\end{document}
