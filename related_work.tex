\section{Related Work} \label{related_work}

In this section, we will give an overview on existing murder mystery video games and examine how they approach the various challenges of the genre, survey projects that have used procedural generation in the context of game narratives and present the rule-specification language Ceptre.

\subsection{Murder Mystery Video Games}

While prevalent in many genres, as seen in section \ref{introduction}, murder mysteries have only partly been explored in video games.

Most murder mystery games rely on handwritten narratives, that are similar in plot to a novel or movie.
The player has the opportunity to follow the detectives along the story and gather evidence, interview suspects and follow leads.
In some cases they may also expose lies and find holes in the presented testimonies.
However, most games lack the ability to make connections between the evidence and testimonies and come to their own conclusions. With most steps represented as multiple choice questions, the player just has to pick the correct choice or guess until they get it right \cite{brown_2017}. 

This tactic is employed in games like LA Noire, where falsely accusing a suspect will make him offer less information to you, but will never have game-changing consequences.
This ensures that the player will always be able to complete the game, they must always reach the next scripted story point in the branching or linear narrative.

While this is a valid choice, it has been noted that the resulting game may be less compelling.
When the answers are all given within the game, the player can not become fully immersed in playing the clever detective \cite{goldstein_2018}.

However, when players are left on their own to collect all evidence, it can lead to games like Murder on the Mississippi, which quickly become insolvable if a player misses a piece of information early on in the game \cite{petit_2018}.

To make sure no clues can be missed while still allowing for more interaction than simple multiple choice options, games often feature in game notebooks or a collection of relevant items and notes.
For example in the game Discworld Noir or...
These written down clues often spoil the puzzles they represent as they need to feature the important aspect of the clue in order to be useful at all.

Some games have attempted to remove this step by letting players take their own notes and offering other option to receive more information, such as the in-game search engine in The Shivah or the telephone book in Sherlock Holmes: Consuting Detective, which when used correctly open up potential new options for interaction.
\todo{Why is this not ideal?}
The game Her Story is an example of a completely different way of approaching this problem.
The murder mystery game does not involve any sort of confrontation of the suspects.
In fact there is no in game way of determining if you have come to the right conclusion at all.
The player can simply gather information by searching a police database until they are satisfied with their understanding of the events.

Reaching the correct solutions in this manner tends to be more satisfying to the player, as it is more of a testament to their skills.

Yet no matter to which of these categories a game belongs, it has a relatively low replay value, as once the murderer is known the main motivation for playing the game ceases to exist.

ClueGen attempts to rectify this by using procedural generation to the plots of murder mysteries.
A number of possible motives and histories are used to create a multitude of narratives.
The addition of possible red herrings or lying non-player characters enhances the complexity even further.
Even with a relatively small set of predefined options the resulting mysteries were sufficiently interesting to the players.
The players recognized complex, underlying intentions and structures, even in cases where these were not intended by the procedural generation program \cite{AIIDE1614070}.

\subsection{Procedural Narrative Generation}

In the GDC Talk \enquote{Best Practices for Procedural Narrative Generation} Rogelio Cardona-Rivera and Chris Martens outline different approaches to procedurally generate stories.
They demonstrate examples of \emph{Ceptre}, a rule language using linear logic to formulate possible events, which we will describe in more detail below.\todo{ref}

Ashlock demonstrated an approach to evolve L-system rules and their interpretation using genetic algorithms to procedurally generate plants.
This approach seems promising to apply on Ceptre, which resembles L-systems in its structure.
The result of each mutation was presented to the user who would intuitively choose plants that looked natural to him or her.
This reveals one challenge when applied to the domain of generative narratives, as grasping whether a generated story makes sense is not as intuitive as judging the looks of a plant.\todo{ref}

For the game PromWeek, McCoy et al created a \enquote{model for socially-oriented gameplay} named \emph{Comme il Faut}.
The model draws from a corpus of social norms that inform the decisions of AI characters each turn of the game.
It then generates little animations to visualize to the player how characters react to his or her actions.
To reach this point, PromWeek required two years of manual authoring to embed social biases and convention in a formalized rule system. \todo{ref}

A similar demonstration of an engaging, fully realized social game is Façade.
Here, as well, it took the authors multiple years to create the necessary rules.
Instead of leaning heavily on exclusively generated content, Facade chose to define story events that may happen at any time if its preconditions are fulfilled, dynamically influencing the story and shaping it as much as possible based on the player's interaction with the game.\todo{ref}

Rowe et al demonstrate the use of reinforcement learning to tweak stories.
A narrative planning system was trained by having a large number of pupils play an educative story-based game and fill out surveys on their experience.
Different branching story options were selected for different players.
This allowed an automated analysis of the story variants that were particularly well received and had the highest impact on the education of the players.\todo{ref}

\subsection{Ceptre} \todo{basically abstract of their paper}

Ceptre\footnote{https://github.com/chrisamaphone/interactive-lp} is a "rule specification language" for games \cite{martens_2015}.
Its intention is to allow rapid prototyping by providing a language specific to this domain.
It uses linear logic as the foundation for the interpretation, which will be explained in more detail in section \ref{linear_logic}.

In Ceptre a "game" consists of terms, predicates and rules, as well as an initial configuration.

The game designer may \todo{essentially repeat abstract and focus on parts important to us}
