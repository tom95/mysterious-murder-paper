\section{Related Work} \label{related_work}

In this section, we will give an overview on existing murder mystery video games and examine how they approach the various challenges of the genre, survey projects that have used procedural generation in the context of game narratives and present the rule-specification language Ceptre created by Martens \cite{martens_2015}, which our own system is based on.

\subsection{Murder Mystery Video Games}

While prevalent in many genres, as seen in Section \ref{introduction}, murder mysteries have only been explored in video games rarely, when compared to other genres.
We put forward the claim that this is due to the complexity involved in detective work as portrayed by literature and film.
Most murder mystery games rely on handwritten narratives that are similar in plot to a novel or movie.
The player has the opportunity to follow the detectives along the story and gather evidence, interview suspects, or follow leads.
In some cases, they may also expose lies and find holes in presented testimonies.
However, most games do not allow the players to make connections between the evidence and testimonies and let them come to their own conclusions. With most steps represented as multiple choice questions, the players just have to pick the correct choice or guess until they get it right \cite{brown_2017}.

This method is employed in games like \emph{LA Noire}, where falsely accusing suspects will make them offer less information to the player, but will never have game-changing consequences.
This ensures that players will always be able to complete the game; they will always be led to reach the next scripted story point in the branching or linear narrative, even if they took wrong decisions before.

While this is a valid choice, it has been noted that the resulting game may be less compelling.
When the answers are all given within the game, the player cannot become fully immersed in playing the clever detective \cite{goldstein_2018}.

However, when players are left on their own to collect all evidence, it can lead to games like \emph{Murder on the Mississippi}, which becomes unsolvable if players miss a piece of information early on in the game \cite{petit_2018}.

To make sure no clues can be missed while still allowing for more interaction than simple multiple choice options, games often feature in-game notebooks or a collection of relevant items and notes.
For example in the game \emph{Discworld Noir} or \emph{Contradiction: Spot the Liar!}.
These written down clues often spoil the puzzles they represent as they need to feature the important aspect of the clue in order to be useful at all.

Some games have attempted to remove this step by letting players take their own notes and offering other option to receive more information, such as the in-game search engine in \emph{The Shivah} or the telephone book in \emph{Sherlock Holmes: Consulting Detective}, which, only when used correctly, open up potential new options for interaction.
While this may increase the challenge and subsequent feeling of reward, it may still convey a feeling of just following in someone else's footsteps.
The game \emph{Her Story} is an example of a completely different way of approaching this problem.
The murder mystery game does not involve any sort of confrontation of the suspects.
In fact, there is no in-game way of determining if you have come to the right conclusion at all.
Players simply gather information by searching a police database until they are satisfied with their understanding of the events.

We claim that reaching the correct solutions in this manner tends to be more satisfying to players, as it is a testament to their skills.

Yet, no matter to which of these categories a game belongs, it has a relatively low replay value, as once the murderer is known, the main motivation for playing the game ceases to exist.

\emph{ClueGen} \cite{stockdale_2016} attempts to rectify this by using procedural generation to the plots of murder mysteries.
A number of possible motives and histories are used to create a multitude of narratives.
The addition of possible red herrings or lying non-player characters enhances the complexity even further.
Even with a relatively small set of predefined options, the resulting mysteries were sufficiently interesting to players.
The players recognized complex, underlying intentions, and structures, even in cases where these were not intended by the author of the rules informing the procedural generation program.

\subsection{Procedural Narrative Generation}

For the game PromWeek, McCoy et al \cite{mccoy_2013, McCoy:2011:CIF:3014589.3014617} created a \enquote{model for socially-oriented gameplay} named \emph{Comme il Faut}.
The model draws from a corpus of social norms that inform the decisions of non-player characters each turn of the game.
It then generates animations to visualize to players how characters react to their decisions.
To reach this point, PromWeek required two years of manual authoring to embed social biases and conventions in a formalized rule system.

A similar demonstration of an engaging, fully realized social game is Fa\c{c}ade \cite{Mateas_2003}.
Here, as well, it took the authors multiple years to create the necessary rules.
Instead of leaning heavily on exclusively generated content, Facade chose to define story events that may happen at any time if its preconditions are fulfilled, dynamically influencing the story and shaping it as much as possible based on the player's interaction with the game.

Rowe et al \cite{rowe_2014} demonstrate the use of reinforcement learning to tweak stories.
A narrative planning system was trained by having a large number of pupils play an educative story-based game and fill out surveys on their experience.
Different branching story options were selected for different players.
This allowed an automated analysis of the story variants that were particularly well received and had the highest impact on the education of the players.

\emph{Ceptre}\footnote{https://github.com/chrisamaphone/interactive-lp} is a \enquote{rule specification language} for interactive narratives or simulations \cite{martens_2015}.
Its intention is to allow rapid prototyping by providing a language specific to this domain.
It uses linear logic as the foundation for the interpretation.

Linear logic is a formal logic that extends the classical logic used in mathematics.
The classical logic concerns itself only with \enquote{stable truths}.
These can, however, not always be applied directly to real-life situations, as they lack causality \cite{girard_1995}.
While in classical logic all conditions stay the same after an evaluation, in reality, the conditions may be modified.
For example, trading money for an object removes the possibility of spending this money again and changes the truths \enquote{I have money} to \enquote{I have this object}.
Linear logic focuses on its formulas as resources, rather than truths \cite{sep-logic-linear}.
New resources may be added to the simulation's state or removed in the course of rule evaluations.

Ceptre allows a game designer to specify terms, rules, and predicates as well as an initial configuration $S$.
The initial configuration $S$ consists of a set of predicates that represent the collection of truths at this point, such as \enquote{I have money}.
The execution of a Ceptre simulation consists of a series of repeated steps.
During each step, it will first be determined which rules can be applied in accord with the current state.
A rule may have a set of preconditions, which is the resources that will be consumed from the state if that rule is applied, and a set of resulting resources.
This corresponds to the possible actions an agent could take or in which a simulation may evolve.

A valid rule can either be chosen interactively by a player or randomly.
The chosen rule will then be applied to the state, creating a new instance of state $S'$, on which the simulation step is then repeated until some terminal condition is reached, for example, no more rules being available.