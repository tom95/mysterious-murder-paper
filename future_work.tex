\section{Future Work} \label{future_work}

While the status of this project currently puts it in a playable state, the final outcome is far from being a finished or fully enjoyable game.
In this section, we will talk about the steps that are necessary before a version of this prototype can be tested by players to a larger extent, such as giving the simulated characters more distinct personalities and the ability to lie, as well as a concept of searching for additional evidence.

\subsection{Enhancing Characters}

The decision to use emojis to represent our characters has served us well, making recognition quick and easy.
The characters themselves are, however, still very rudimentary and lack personality.

We would like to add more options to the characters themselves, such as age and occupation, as well as use the already defined traits during the investigation stage. Trusting characters could reveal information more easily, while others may try to keep secrets.

Using the emojis in a less random and more thought out way would allow us to make relationships between characters more clear. Characters of one family could, for example, have the same skin color (in most cases).

Currently, we only generate whether two characters are related, we do not specify how.
This might be an area where we could improve, which would allow us to introduce more complex rules and allow for more interesting stories, such as rivaling siblings. To make character relations more obvious, we would like to change their last names, so that families and married couples have the same name.

\subsection{Enhancing the Investigation Stage}
To make the game satisfying to players, they should have the option to uncover hidden motives.
At the moment, none of the actors will attempt to lie to the player, they will not even try to hide information.
While not all actors possess knowledge of everything that has happened, it is only a matter of dedication for the player to question all actors before eventually finding the actor who admits to the murder.

To remedy this situation, we propose a gameplay mechanic, wherein local knowledge exists and certain actors can lie.
More specifically, each actor is allowed to omit information that they consider sensitive.
Only the murderer is allowed to also falsify statements.
For any rule that is taken, uninvolved actors have a chance to witness this action happening.
The probability that this happens is defined per rule, to prevent sensitive information from being witnessed too frequently, such that more realistic testimonies come together.
With these mechanics in place, the game's core objective for players will be to identify the one actor whose testimony contains actual false statements, thereby revealing themselves as the murderer.
This is challenged by the fact that actors the murderer interacted with may omit this interaction, or that the murderer primarily interacted with the victim and no other actors witnessed these interactions.

In a further step, accomplices to the murderer may be defined based on relationships.
The relationship between the murderer and the accomplice may be hidden, the accomplice may be a secret lover for example, or open through family connections or the like.
The accomplice would then also get the option to falsify information in order to cover both of them.
In the hopes that there will be contradictions between the testimony of the murderer and their accomplice, the player would get an additional challenge to find all those involved.

With this model, it is, however, questionable if the player will always be able to solve the mystery.
It may very well happen that simply not enough information from uninvolved actors is presented to find contradictions and identify the murderer.
While this, of course, reflects real-life murder mysteries well, it will not lead to satisfying gameplay.
As such, it would be desirable to automatically detect how many statements contradict to filter out potentially unsolvable mysteries.

The current investigation stage could be further improved by not making all actions stop immediately after the murder.
This would make it possible for actors to rid themselves of the murder weapon or grieve for the victim.
Ideally, actors would be able to blame the murder on other characters to create scenarios that are more difficult to unravel.

Once these points have been realized, rigorous playtesting would be the next step.
In particular, the number of characters will provide an interesting point of balance.
Having more characters increases the load of information for the player, but also provides them with more points of reference for figuring out contradictions in a testimony and potentially a richer story with more complex character relationships.
Since the interactions between actors are the only way the player gets to know them, giving actors the option to interact more will lead to players identifying actors in richer ways, as already shown in ClueGen \cite{stockdale_2016}.
An actor may become a lover of particularly many other actors or a family feud may arise.

It also stands to investigate if we can add a mechanic that would create red herrings, or if the procedural generation already lets the player find red herrings without requiring explicit authorship of such rules.
